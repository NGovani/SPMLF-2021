\DeclareUnicodeCharacter{2212}{-}
\documentclass[11pt]{article}

    \usepackage[utf8]{inputenc}

    \usepackage[breakable]{tcolorbox}
    \usepackage{parskip} % Stop auto-indenting (to mimic markdown behaviour)
    
    \usepackage{iftex}
    \ifPDFTeX
    	\usepackage[T1]{fontenc}
    	\usepackage{mathpazo}
    \else
    	\usepackage{fontspec}
    \fi

    % Basic figure setup, for now with no caption control since it's done
    % automatically by Pandoc (which extracts ![](path) syntax from Markdown).
    \usepackage{graphicx}
    % Maintain compatibility with old templates. Remove in nbconvert 6.0
    \let\Oldincludegraphics\includegraphics
    % Ensure that by default, figures have no caption (until we provide a
    % proper Figure object with a Caption API and a way to capture that
    % in the conversion process - todo).
    \usepackage{caption}
    \DeclareCaptionFormat{nocaption}{}
    \captionsetup{format=nocaption,aboveskip=0pt,belowskip=0pt}

    \usepackage{float}
    \floatplacement{figure}{H} % forces figures to be placed at the correct location
    \usepackage{xcolor} % Allow colors to be defined
    \usepackage{enumerate} % Needed for markdown enumerations to work
    \usepackage{geometry} % Used to adjust the document margins
    \usepackage{amsmath} % Equations
    \usepackage{amssymb} % Equations
    \usepackage{textcomp} % defines textquotesingle
    % Hack from http://tex.stackexchange.com/a/47451/13684:
    \AtBeginDocument{%
        \def\PYZsq{\textquotesingle}% Upright quotes in Pygmentized code
    }
    \usepackage{upquote} % Upright quotes for verbatim code
    \usepackage{eurosym} % defines \euro
    \usepackage[mathletters]{ucs} % Extended unicode (utf-8) support
    \usepackage{fancyvrb} % verbatim replacement that allows latex
    \usepackage{grffile} % extends the file name processing of package graphics 
                         % to support a larger range
    \makeatletter % fix for old versions of grffile with XeLaTeX
    \@ifpackagelater{grffile}{2019/11/01}
    {
      % Do nothing on new versions
    }
    {
      \def\Gread@@xetex#1{%
        \IfFileExists{"\Gin@base".bb}%
        {\Gread@eps{\Gin@base.bb}}%
        {\Gread@@xetex@aux#1}%
      }
    }
    \makeatother
    \usepackage[Export]{adjustbox} % Used to constrain images to a maximum size
    \adjustboxset{max size={0.9\linewidth}{0.9\paperheight}}

    % The hyperref package gives us a pdf with properly built
    % internal navigation ('pdf bookmarks' for the table of contents,
    % internal cross-reference links, web links for URLs, etc.)
    \usepackage{hyperref}
    % The default LaTeX title has an obnoxious amount of whitespace. By default,
    % titling removes some of it. It also provides customization options.
    \usepackage{titling}
    \usepackage{longtable} % longtable support required by pandoc >1.10
    \usepackage{booktabs}  % table support for pandoc > 1.12.2
    \usepackage[inline]{enumitem} % IRkernel/repr support (it uses the enumerate* environment)
    \usepackage[normalem]{ulem} % ulem is needed to support strikethroughs (\sout)
                                % normalem makes italics be italics, not underlines
    \usepackage{mathrsfs}
    

    
    % Colors for the hyperref package
    \definecolor{urlcolor}{rgb}{0,.145,.698}
    \definecolor{linkcolor}{rgb}{.71,0.21,0.01}
    \definecolor{citecolor}{rgb}{.12,.54,.11}

    % ANSI colors
    \definecolor{ansi-black}{HTML}{3E424D}
    \definecolor{ansi-black-intense}{HTML}{282C36}
    \definecolor{ansi-red}{HTML}{E75C58}
    \definecolor{ansi-red-intense}{HTML}{B22B31}
    \definecolor{ansi-green}{HTML}{00A250}
    \definecolor{ansi-green-intense}{HTML}{007427}
    \definecolor{ansi-yellow}{HTML}{DDB62B}
    \definecolor{ansi-yellow-intense}{HTML}{B27D12}
    \definecolor{ansi-blue}{HTML}{208FFB}
    \definecolor{ansi-blue-intense}{HTML}{0065CA}
    \definecolor{ansi-magenta}{HTML}{D160C4}
    \definecolor{ansi-magenta-intense}{HTML}{A03196}
    \definecolor{ansi-cyan}{HTML}{60C6C8}
    \definecolor{ansi-cyan-intense}{HTML}{258F8F}
    \definecolor{ansi-white}{HTML}{C5C1B4}
    \definecolor{ansi-white-intense}{HTML}{A1A6B2}
    \definecolor{ansi-default-inverse-fg}{HTML}{FFFFFF}
    \definecolor{ansi-default-inverse-bg}{HTML}{000000}

    % common color for the border for error outputs.
    \definecolor{outerrorbackground}{HTML}{FFDFDF}

    % commands and environments needed by pandoc snippets
    % extracted from the output of `pandoc -s`
    \providecommand{\tightlist}{%
      \setlength{\itemsep}{0pt}\setlength{\parskip}{0pt}}
    \DefineVerbatimEnvironment{Highlighting}{Verbatim}{commandchars=\\\{\}}
    % Add ',fontsize=\small' for more characters per line
    \newenvironment{Shaded}{}{}
    \newcommand{\KeywordTok}[1]{\textcolor[rgb]{0.00,0.44,0.13}{\textbf{{#1}}}}
    \newcommand{\DataTypeTok}[1]{\textcolor[rgb]{0.56,0.13,0.00}{{#1}}}
    \newcommand{\DecValTok}[1]{\textcolor[rgb]{0.25,0.63,0.44}{{#1}}}
    \newcommand{\BaseNTok}[1]{\textcolor[rgb]{0.25,0.63,0.44}{{#1}}}
    \newcommand{\FloatTok}[1]{\textcolor[rgb]{0.25,0.63,0.44}{{#1}}}
    \newcommand{\CharTok}[1]{\textcolor[rgb]{0.25,0.44,0.63}{{#1}}}
    \newcommand{\StringTok}[1]{\textcolor[rgb]{0.25,0.44,0.63}{{#1}}}
    \newcommand{\CommentTok}[1]{\textcolor[rgb]{0.38,0.63,0.69}{\textit{{#1}}}}
    \newcommand{\OtherTok}[1]{\textcolor[rgb]{0.00,0.44,0.13}{{#1}}}
    \newcommand{\AlertTok}[1]{\textcolor[rgb]{1.00,0.00,0.00}{\textbf{{#1}}}}
    \newcommand{\FunctionTok}[1]{\textcolor[rgb]{0.02,0.16,0.49}{{#1}}}
    \newcommand{\RegionMarkerTok}[1]{{#1}}
    \newcommand{\ErrorTok}[1]{\textcolor[rgb]{1.00,0.00,0.00}{\textbf{{#1}}}}
    \newcommand{\NormalTok}[1]{{#1}}
    
    % Additional commands for more recent versions of Pandoc
    \newcommand{\ConstantTok}[1]{\textcolor[rgb]{0.53,0.00,0.00}{{#1}}}
    \newcommand{\SpecialCharTok}[1]{\textcolor[rgb]{0.25,0.44,0.63}{{#1}}}
    \newcommand{\VerbatimStringTok}[1]{\textcolor[rgb]{0.25,0.44,0.63}{{#1}}}
    \newcommand{\SpecialStringTok}[1]{\textcolor[rgb]{0.73,0.40,0.53}{{#1}}}
    \newcommand{\ImportTok}[1]{{#1}}
    \newcommand{\DocumentationTok}[1]{\textcolor[rgb]{0.73,0.13,0.13}{\textit{{#1}}}}
    \newcommand{\AnnotationTok}[1]{\textcolor[rgb]{0.38,0.63,0.69}{\textbf{\textit{{#1}}}}}
    \newcommand{\CommentVarTok}[1]{\textcolor[rgb]{0.38,0.63,0.69}{\textbf{\textit{{#1}}}}}
    \newcommand{\VariableTok}[1]{\textcolor[rgb]{0.10,0.09,0.49}{{#1}}}
    \newcommand{\ControlFlowTok}[1]{\textcolor[rgb]{0.00,0.44,0.13}{\textbf{{#1}}}}
    \newcommand{\OperatorTok}[1]{\textcolor[rgb]{0.40,0.40,0.40}{{#1}}}
    \newcommand{\BuiltInTok}[1]{{#1}}
    \newcommand{\ExtensionTok}[1]{{#1}}
    \newcommand{\PreprocessorTok}[1]{\textcolor[rgb]{0.74,0.48,0.00}{{#1}}}
    \newcommand{\AttributeTok}[1]{\textcolor[rgb]{0.49,0.56,0.16}{{#1}}}
    \newcommand{\InformationTok}[1]{\textcolor[rgb]{0.38,0.63,0.69}{\textbf{\textit{{#1}}}}}
    \newcommand{\WarningTok}[1]{\textcolor[rgb]{0.38,0.63,0.69}{\textbf{\textit{{#1}}}}}
    
    
    % Define a nice break command that doesn't care if a line doesn't already
    % exist.
    \def\br{\hspace*{\fill} \\* }
    % Math Jax compatibility definitions
    \def\gt{>}
    \def\lt{<}
    \let\Oldtex\TeX
    \let\Oldlatex\LaTeX
    \renewcommand{\TeX}{\textrm{\Oldtex}}
    \renewcommand{\LaTeX}{\textrm{\Oldlatex}}
    % Document parameters
    % Document title
    \title{SPMLF CW 20-21}
    
    
    
    
    
% Pygments definitions
\makeatletter
\def\PY@reset{\let\PY@it=\relax \let\PY@bf=\relax%
    \let\PY@ul=\relax \let\PY@tc=\relax%
    \let\PY@bc=\relax \let\PY@ff=\relax}
\def\PY@tok#1{\csname PY@tok@#1\endcsname}
\def\PY@toks#1+{\ifx\relax#1\empty\else%
    \PY@tok{#1}\expandafter\PY@toks\fi}
\def\PY@do#1{\PY@bc{\PY@tc{\PY@ul{%
    \PY@it{\PY@bf{\PY@ff{#1}}}}}}}
\def\PY#1#2{\PY@reset\PY@toks#1+\relax+\PY@do{#2}}

\expandafter\def\csname PY@tok@w\endcsname{\def\PY@tc##1{\textcolor[rgb]{0.73,0.73,0.73}{##1}}}
\expandafter\def\csname PY@tok@c\endcsname{\let\PY@it=\textit\def\PY@tc##1{\textcolor[rgb]{0.25,0.50,0.50}{##1}}}
\expandafter\def\csname PY@tok@cp\endcsname{\def\PY@tc##1{\textcolor[rgb]{0.74,0.48,0.00}{##1}}}
\expandafter\def\csname PY@tok@k\endcsname{\let\PY@bf=\textbf\def\PY@tc##1{\textcolor[rgb]{0.00,0.50,0.00}{##1}}}
\expandafter\def\csname PY@tok@kp\endcsname{\def\PY@tc##1{\textcolor[rgb]{0.00,0.50,0.00}{##1}}}
\expandafter\def\csname PY@tok@kt\endcsname{\def\PY@tc##1{\textcolor[rgb]{0.69,0.00,0.25}{##1}}}
\expandafter\def\csname PY@tok@o\endcsname{\def\PY@tc##1{\textcolor[rgb]{0.40,0.40,0.40}{##1}}}
\expandafter\def\csname PY@tok@ow\endcsname{\let\PY@bf=\textbf\def\PY@tc##1{\textcolor[rgb]{0.67,0.13,1.00}{##1}}}
\expandafter\def\csname PY@tok@nb\endcsname{\def\PY@tc##1{\textcolor[rgb]{0.00,0.50,0.00}{##1}}}
\expandafter\def\csname PY@tok@nf\endcsname{\def\PY@tc##1{\textcolor[rgb]{0.00,0.00,1.00}{##1}}}
\expandafter\def\csname PY@tok@nc\endcsname{\let\PY@bf=\textbf\def\PY@tc##1{\textcolor[rgb]{0.00,0.00,1.00}{##1}}}
\expandafter\def\csname PY@tok@nn\endcsname{\let\PY@bf=\textbf\def\PY@tc##1{\textcolor[rgb]{0.00,0.00,1.00}{##1}}}
\expandafter\def\csname PY@tok@ne\endcsname{\let\PY@bf=\textbf\def\PY@tc##1{\textcolor[rgb]{0.82,0.25,0.23}{##1}}}
\expandafter\def\csname PY@tok@nv\endcsname{\def\PY@tc##1{\textcolor[rgb]{0.10,0.09,0.49}{##1}}}
\expandafter\def\csname PY@tok@no\endcsname{\def\PY@tc##1{\textcolor[rgb]{0.53,0.00,0.00}{##1}}}
\expandafter\def\csname PY@tok@nl\endcsname{\def\PY@tc##1{\textcolor[rgb]{0.63,0.63,0.00}{##1}}}
\expandafter\def\csname PY@tok@ni\endcsname{\let\PY@bf=\textbf\def\PY@tc##1{\textcolor[rgb]{0.60,0.60,0.60}{##1}}}
\expandafter\def\csname PY@tok@na\endcsname{\def\PY@tc##1{\textcolor[rgb]{0.49,0.56,0.16}{##1}}}
\expandafter\def\csname PY@tok@nt\endcsname{\let\PY@bf=\textbf\def\PY@tc##1{\textcolor[rgb]{0.00,0.50,0.00}{##1}}}
\expandafter\def\csname PY@tok@nd\endcsname{\def\PY@tc##1{\textcolor[rgb]{0.67,0.13,1.00}{##1}}}
\expandafter\def\csname PY@tok@s\endcsname{\def\PY@tc##1{\textcolor[rgb]{0.73,0.13,0.13}{##1}}}
\expandafter\def\csname PY@tok@sd\endcsname{\let\PY@it=\textit\def\PY@tc##1{\textcolor[rgb]{0.73,0.13,0.13}{##1}}}
\expandafter\def\csname PY@tok@si\endcsname{\let\PY@bf=\textbf\def\PY@tc##1{\textcolor[rgb]{0.73,0.40,0.53}{##1}}}
\expandafter\def\csname PY@tok@se\endcsname{\let\PY@bf=\textbf\def\PY@tc##1{\textcolor[rgb]{0.73,0.40,0.13}{##1}}}
\expandafter\def\csname PY@tok@sr\endcsname{\def\PY@tc##1{\textcolor[rgb]{0.73,0.40,0.53}{##1}}}
\expandafter\def\csname PY@tok@ss\endcsname{\def\PY@tc##1{\textcolor[rgb]{0.10,0.09,0.49}{##1}}}
\expandafter\def\csname PY@tok@sx\endcsname{\def\PY@tc##1{\textcolor[rgb]{0.00,0.50,0.00}{##1}}}
\expandafter\def\csname PY@tok@m\endcsname{\def\PY@tc##1{\textcolor[rgb]{0.40,0.40,0.40}{##1}}}
\expandafter\def\csname PY@tok@gh\endcsname{\let\PY@bf=\textbf\def\PY@tc##1{\textcolor[rgb]{0.00,0.00,0.50}{##1}}}
\expandafter\def\csname PY@tok@gu\endcsname{\let\PY@bf=\textbf\def\PY@tc##1{\textcolor[rgb]{0.50,0.00,0.50}{##1}}}
\expandafter\def\csname PY@tok@gd\endcsname{\def\PY@tc##1{\textcolor[rgb]{0.63,0.00,0.00}{##1}}}
\expandafter\def\csname PY@tok@gi\endcsname{\def\PY@tc##1{\textcolor[rgb]{0.00,0.63,0.00}{##1}}}
\expandafter\def\csname PY@tok@gr\endcsname{\def\PY@tc##1{\textcolor[rgb]{1.00,0.00,0.00}{##1}}}
\expandafter\def\csname PY@tok@ge\endcsname{\let\PY@it=\textit}
\expandafter\def\csname PY@tok@gs\endcsname{\let\PY@bf=\textbf}
\expandafter\def\csname PY@tok@gp\endcsname{\let\PY@bf=\textbf\def\PY@tc##1{\textcolor[rgb]{0.00,0.00,0.50}{##1}}}
\expandafter\def\csname PY@tok@go\endcsname{\def\PY@tc##1{\textcolor[rgb]{0.53,0.53,0.53}{##1}}}
\expandafter\def\csname PY@tok@gt\endcsname{\def\PY@tc##1{\textcolor[rgb]{0.00,0.27,0.87}{##1}}}
\expandafter\def\csname PY@tok@err\endcsname{\def\PY@bc##1{\setlength{\fboxsep}{0pt}\fcolorbox[rgb]{1.00,0.00,0.00}{1,1,1}{\strut ##1}}}
\expandafter\def\csname PY@tok@kc\endcsname{\let\PY@bf=\textbf\def\PY@tc##1{\textcolor[rgb]{0.00,0.50,0.00}{##1}}}
\expandafter\def\csname PY@tok@kd\endcsname{\let\PY@bf=\textbf\def\PY@tc##1{\textcolor[rgb]{0.00,0.50,0.00}{##1}}}
\expandafter\def\csname PY@tok@kn\endcsname{\let\PY@bf=\textbf\def\PY@tc##1{\textcolor[rgb]{0.00,0.50,0.00}{##1}}}
\expandafter\def\csname PY@tok@kr\endcsname{\let\PY@bf=\textbf\def\PY@tc##1{\textcolor[rgb]{0.00,0.50,0.00}{##1}}}
\expandafter\def\csname PY@tok@bp\endcsname{\def\PY@tc##1{\textcolor[rgb]{0.00,0.50,0.00}{##1}}}
\expandafter\def\csname PY@tok@fm\endcsname{\def\PY@tc##1{\textcolor[rgb]{0.00,0.00,1.00}{##1}}}
\expandafter\def\csname PY@tok@vc\endcsname{\def\PY@tc##1{\textcolor[rgb]{0.10,0.09,0.49}{##1}}}
\expandafter\def\csname PY@tok@vg\endcsname{\def\PY@tc##1{\textcolor[rgb]{0.10,0.09,0.49}{##1}}}
\expandafter\def\csname PY@tok@vi\endcsname{\def\PY@tc##1{\textcolor[rgb]{0.10,0.09,0.49}{##1}}}
\expandafter\def\csname PY@tok@vm\endcsname{\def\PY@tc##1{\textcolor[rgb]{0.10,0.09,0.49}{##1}}}
\expandafter\def\csname PY@tok@sa\endcsname{\def\PY@tc##1{\textcolor[rgb]{0.73,0.13,0.13}{##1}}}
\expandafter\def\csname PY@tok@sb\endcsname{\def\PY@tc##1{\textcolor[rgb]{0.73,0.13,0.13}{##1}}}
\expandafter\def\csname PY@tok@sc\endcsname{\def\PY@tc##1{\textcolor[rgb]{0.73,0.13,0.13}{##1}}}
\expandafter\def\csname PY@tok@dl\endcsname{\def\PY@tc##1{\textcolor[rgb]{0.73,0.13,0.13}{##1}}}
\expandafter\def\csname PY@tok@s2\endcsname{\def\PY@tc##1{\textcolor[rgb]{0.73,0.13,0.13}{##1}}}
\expandafter\def\csname PY@tok@sh\endcsname{\def\PY@tc##1{\textcolor[rgb]{0.73,0.13,0.13}{##1}}}
\expandafter\def\csname PY@tok@s1\endcsname{\def\PY@tc##1{\textcolor[rgb]{0.73,0.13,0.13}{##1}}}
\expandafter\def\csname PY@tok@mb\endcsname{\def\PY@tc##1{\textcolor[rgb]{0.40,0.40,0.40}{##1}}}
\expandafter\def\csname PY@tok@mf\endcsname{\def\PY@tc##1{\textcolor[rgb]{0.40,0.40,0.40}{##1}}}
\expandafter\def\csname PY@tok@mh\endcsname{\def\PY@tc##1{\textcolor[rgb]{0.40,0.40,0.40}{##1}}}
\expandafter\def\csname PY@tok@mi\endcsname{\def\PY@tc##1{\textcolor[rgb]{0.40,0.40,0.40}{##1}}}
\expandafter\def\csname PY@tok@il\endcsname{\def\PY@tc##1{\textcolor[rgb]{0.40,0.40,0.40}{##1}}}
\expandafter\def\csname PY@tok@mo\endcsname{\def\PY@tc##1{\textcolor[rgb]{0.40,0.40,0.40}{##1}}}
\expandafter\def\csname PY@tok@ch\endcsname{\let\PY@it=\textit\def\PY@tc##1{\textcolor[rgb]{0.25,0.50,0.50}{##1}}}
\expandafter\def\csname PY@tok@cm\endcsname{\let\PY@it=\textit\def\PY@tc##1{\textcolor[rgb]{0.25,0.50,0.50}{##1}}}
\expandafter\def\csname PY@tok@cpf\endcsname{\let\PY@it=\textit\def\PY@tc##1{\textcolor[rgb]{0.25,0.50,0.50}{##1}}}
\expandafter\def\csname PY@tok@c1\endcsname{\let\PY@it=\textit\def\PY@tc##1{\textcolor[rgb]{0.25,0.50,0.50}{##1}}}
\expandafter\def\csname PY@tok@cs\endcsname{\let\PY@it=\textit\def\PY@tc##1{\textcolor[rgb]{0.25,0.50,0.50}{##1}}}

\def\PYZbs{\char`\\}
\def\PYZus{\char`\_}
\def\PYZob{\char`\{}
\def\PYZcb{\char`\}}
\def\PYZca{\char`\^}
\def\PYZam{\char`\&}
\def\PYZlt{\char`\<}
\def\PYZgt{\char`\>}
\def\PYZsh{\char`\#}
\def\PYZpc{\char`\%}
\def\PYZdl{\char`\$}
\def\PYZhy{\char`\-}
\def\PYZsq{\char`\'}
\def\PYZdq{\char`\"}
\def\PYZti{\char`\~}
% for compatibility with earlier versions
\def\PYZat{@}
\def\PYZlb{[}
\def\PYZrb{]}
\makeatother


    % For linebreaks inside Verbatim environment from package fancyvrb. 
    \makeatletter
        \newbox\Wrappedcontinuationbox 
        \newbox\Wrappedvisiblespacebox 
        \newcommand*\Wrappedvisiblespace {\textcolor{red}{\textvisiblespace}} 
        \newcommand*\Wrappedcontinuationsymbol {\textcolor{red}{\llap{\tiny$\m@th\hookrightarrow$}}} 
        \newcommand*\Wrappedcontinuationindent {3ex } 
        \newcommand*\Wrappedafterbreak {\kern\Wrappedcontinuationindent\copy\Wrappedcontinuationbox} 
        % Take advantage of the already applied Pygments mark-up to insert 
        % potential linebreaks for TeX processing. 
        %        {, <, #, %, $, ' and ": go to next line. 
        %        _, }, ^, &, >, - and ~: stay at end of broken line. 
        % Use of \textquotesingle for straight quote. 
        \newcommand*\Wrappedbreaksatspecials {% 
            \def\PYGZus{\discretionary{\char`\_}{\Wrappedafterbreak}{\char`\_}}% 
            \def\PYGZob{\discretionary{}{\Wrappedafterbreak\char`\{}{\char`\{}}% 
            \def\PYGZcb{\discretionary{\char`\}}{\Wrappedafterbreak}{\char`\}}}% 
            \def\PYGZca{\discretionary{\char`\^}{\Wrappedafterbreak}{\char`\^}}% 
            \def\PYGZam{\discretionary{\char`\&}{\Wrappedafterbreak}{\char`\&}}% 
            \def\PYGZlt{\discretionary{}{\Wrappedafterbreak\char`\<}{\char`\<}}% 
            \def\PYGZgt{\discretionary{\char`\>}{\Wrappedafterbreak}{\char`\>}}% 
            \def\PYGZsh{\discretionary{}{\Wrappedafterbreak\char`\#}{\char`\#}}% 
            \def\PYGZpc{\discretionary{}{\Wrappedafterbreak\char`\%}{\char`\%}}% 
            \def\PYGZdl{\discretionary{}{\Wrappedafterbreak\char`\$}{\char`\$}}% 
            \def\PYGZhy{\discretionary{\char`\-}{\Wrappedafterbreak}{\char`\-}}% 
            \def\PYGZsq{\discretionary{}{\Wrappedafterbreak\textquotesingle}{\textquotesingle}}% 
            \def\PYGZdq{\discretionary{}{\Wrappedafterbreak\char`\"}{\char`\"}}% 
            \def\PYGZti{\discretionary{\char`\~}{\Wrappedafterbreak}{\char`\~}}% 
        } 
        % Some characters . , ; ? ! / are not pygmentized. 
        % This macro makes them "active" and they will insert potential linebreaks 
        \newcommand*\Wrappedbreaksatpunct {% 
            \lccode`\~`\.\lowercase{\def~}{\discretionary{\hbox{\char`\.}}{\Wrappedafterbreak}{\hbox{\char`\.}}}% 
            \lccode`\~`\,\lowercase{\def~}{\discretionary{\hbox{\char`\,}}{\Wrappedafterbreak}{\hbox{\char`\,}}}% 
            \lccode`\~`\;\lowercase{\def~}{\discretionary{\hbox{\char`\;}}{\Wrappedafterbreak}{\hbox{\char`\;}}}% 
            \lccode`\~`\:\lowercase{\def~}{\discretionary{\hbox{\char`\:}}{\Wrappedafterbreak}{\hbox{\char`\:}}}% 
            \lccode`\~`\?\lowercase{\def~}{\discretionary{\hbox{\char`\?}}{\Wrappedafterbreak}{\hbox{\char`\?}}}% 
            \lccode`\~`\!\lowercase{\def~}{\discretionary{\hbox{\char`\!}}{\Wrappedafterbreak}{\hbox{\char`\!}}}% 
            \lccode`\~`\/\lowercase{\def~}{\discretionary{\hbox{\char`\/}}{\Wrappedafterbreak}{\hbox{\char`\/}}}% 
            \catcode`\.\active
            \catcode`\,\active 
            \catcode`\;\active
            \catcode`\:\active
            \catcode`\?\active
            \catcode`\!\active
            \catcode`\/\active 
            \lccode`\~`\~ 	
        }
    \makeatother

    \let\OriginalVerbatim=\Verbatim
    \makeatletter
    \renewcommand{\Verbatim}[1][1]{%
        %\parskip\z@skip
        \sbox\Wrappedcontinuationbox {\Wrappedcontinuationsymbol}%
        \sbox\Wrappedvisiblespacebox {\FV@SetupFont\Wrappedvisiblespace}%
        \def\FancyVerbFormatLine ##1{\hsize\linewidth
            \vtop{\raggedright\hyphenpenalty\z@\exhyphenpenalty\z@
                \doublehyphendemerits\z@\finalhyphendemerits\z@
                \strut ##1\strut}%
        }%
        % If the linebreak is at a space, the latter will be displayed as visible
        % space at end of first line, and a continuation symbol starts next line.
        % Stretch/shrink are however usually zero for typewriter font.
        \def\FV@Space {%
            \nobreak\hskip\z@ plus\fontdimen3\font minus\fontdimen4\font
            \discretionary{\copy\Wrappedvisiblespacebox}{\Wrappedafterbreak}
            {\kern\fontdimen2\font}%
        }%
        
        % Allow breaks at special characters using \PYG... macros.
        \Wrappedbreaksatspecials
        % Breaks at punctuation characters . , ; ? ! and / need catcode=\active 	
        \OriginalVerbatim[#1,codes*=\Wrappedbreaksatpunct]%
    }
    \makeatother

    % Exact colors from NB
    \definecolor{incolor}{HTML}{303F9F}
    \definecolor{outcolor}{HTML}{D84315}
    \definecolor{cellborder}{HTML}{CFCFCF}
    \definecolor{cellbackground}{HTML}{F7F7F7}
    
    % prompt
    \makeatletter
    \newcommand{\boxspacing}{\kern\kvtcb@left@rule\kern\kvtcb@boxsep}
    \makeatother
    \newcommand{\prompt}[4]{
        {\ttfamily\llap{{\color{#2}[#3]:\hspace{3pt}#4}}\vspace{-\baselineskip}}
    }
    

    
    % Prevent overflowing lines due to hard-to-break entities
    \sloppy 
    % Setup hyperref package
    \hypersetup{
      breaklinks=true,  % so long urls are correctly broken across lines
      colorlinks=true,
      urlcolor=urlcolor,
      linkcolor=linkcolor,
      citecolor=citecolor,
      }
    % Slightly bigger margins than the latex defaults
    
    \geometry{verbose,tmargin=1in,bmargin=1in,lmargin=1in,rmargin=1in}
    
    

\begin{document}
    \author{Naim Govani}
    \maketitle
    
    

    
    \hypertarget{regression-methods}{%
\section{1. Regression Methods}\label{regression-methods}}

\hypertarget{processing-stock-price-data-in-python}{%
\subsection{1.1 Processing stock price data in
Python}\label{processing-stock-price-data-in-python}}

\hypertarget{price-log-transform}{%
\subsubsection{1.1.1 Price Log Transform}\label{price-log-transform}}

    The graphs below show the SPX index since the 1930s, the left shows the
absolute value while the right shows the log value

    \begin{center}
    \adjustimage{max size={0.9\linewidth}{0.9\paperheight}}{SPMLF_files/SPMLF_4_0.pdf}
    \end{center}
    { \hspace*{\fill} \\}
    
    \hypertarget{sliding-window-price-time-series}{%
\subsubsection{1.1.2 Sliding window price
time-series}\label{sliding-window-price-time-series}}

\hypertarget{mean}{%
\paragraph{Mean}\label{mean}}

The graphs below show the sliding window time series for the SPX index.
The graph on the left is the absolute value and has no stationarity as
its mean grows exponentially from the 1930s to the 1990s. An argument
could be made that it became somewhat stationary between the late 1990s
and the mid 2010s however the variance of such a distribution would be
too high to make any meaningful conclusions.

The log graph is also non-stationary, however its rise in mean is much
smoother than the alternative. The segment between the late 1990s
and mid 2010s could also been seen as stationary with a much lower
variance than before.

    \begin{center}
    \adjustimage{max size={0.9\linewidth}{0.9\paperheight}}{SPMLF_files/SPMLF_6_0.pdf}
    \end{center}
    { \hspace*{\fill} \\}
    
    \hypertarget{standard-deviation}{%
\paragraph{Standard Deviation}\label{standard-deviation2}}

The standard deviation graphs below show much more stationarity than the
mean graphs, especially the log SPX index. Said, graph seems to have a
constant means around 0.75 with a large variance. The absolute graph,
shown on the left, has an increasing mean making it essentially
non-stationary.

    \begin{center}
    \adjustimage{max size={0.9\linewidth}{0.9\paperheight}}{SPMLF_files/SPMLF_8_0.pdf}
    \end{center}
    { \hspace*{\fill} \\}
    
    \hypertarget{simple-and-log-returns}{%
\subsubsection{1.1.3 Simple and Log
Returns}\label{simple-and-log-returns}}

Below are the simple and log returns of the SPX index.

    \begin{center}
    \adjustimage{max size={0.9\linewidth}{0.9\paperheight}}{SPMLF_files/SPMLF_10_0.pdf}
    \end{center}
    { \hspace*{\fill} \\}
    
    \hypertarget{sliding-window-mean}{%
\paragraph{Sliding Window: Mean}\label{sliding-window-mean}}

As before, we take a sliding window of size 252, of the data and plot
the mean and s.d. The graphs shown below represent the mean of the
sliding window for both simple and log returns. As can be seen, the mean
of both graphs is much more constant meaning that they have more
stationarity than the price graphs shown before. They are also much more
similar to each other than the log price and normal price graphs were.

    \begin{center}
    \adjustimage{max size={0.9\linewidth}{0.9\paperheight}}{SPMLF_files/SPMLF_12_0.pdf}
    \end{center}
    { \hspace*{\fill} \\}
    
    \hypertarget{standard-deviation}{%
\paragraph{Standard Deviation}\label{standard-deviation}}

As with the mean, the standard deviation plots of the returns show more
stationarity than the price plots.

    \begin{center}
    \adjustimage{max size={0.9\linewidth}{0.9\paperheight}}{SPMLF_files/SPMLF_14_0.pdf}
    \end{center}
    { \hspace*{\fill} \\}
    
    \hypertarget{advantages-of-log-returns}{%
\subsubsection{1.1.4 Advantages of Log
Returns}\label{advantages-of-log-returns}}

If we assume prices in the short term are distributed log-normally, then
the log return is normally distributed. This is advantageous because
many statistcal and signal processing methods assume Gaussianity. Log
returns are also time-additive which means that when calculating the
compounding return of a sequence of trades the calculation becomes much
simpler. If we use the identity: $ log(1+r_i) = log(\frac{p_i}{p_{i-1}}) = log(p_i) - log(p_{i-1})$, 
then we can
simplify the as such: 
$\sum_i^n log(1+r_i) = log(1+r_1) + log(1+r_2) + \ldots{} + log(1+r_n) = log(p_n) - log(p_0) $.

To confirm the Gaussianity of data we may use the Jarque-Bera test,
which returns a statistic. The closer this statistic, JB, is to zero the
more normal the distribution of data points is. The graph below shows
that as more data points are introduced the distribution becomes less
Gaussian which is to be expected as the mean of the SPX index is
constantly increasing.

    \begin{center}
    \adjustimage{max size={0.9\linewidth}{0.9\paperheight}}{SPMLF_files/SPMLF_17_0.pdf}
    \end{center}
    { \hspace*{\fill} \\}
    
    \hypertarget{simple-vs-log-return-calculation}{%
\subsubsection{1.1.5 Simple vs Log Return:
Calculation}\label{simple-vs-log-return-calculation}}

Given the example, that if we were to purchase a stock for £1 and the
next day the value goes up to £2 and the day after it goes back to £1 we
can calculate the simple and log returns over the two days. The simple
returns would be {[}1, -0.5{]} (1st element is the returns for day 1,
2nd element is returns for day 2). The log returns is {[}0.69, -0.69{]}.
By summing the log returns we get 0, which tells us that the value of
the stock hasn't changed over two days which is more information than
the simple returns.

\hypertarget{simple-vs-log-return-simple-returns-adv.}{%
\subsubsection{1.1.6 Simple vs Log Return: Simple Returns
Adv.}\label{simple-vs-log-return-simple-returns-adv.}}

Since log-normality of data points only hold for short time periods we
can only use log returns for short-term analysis. Additionally, while
log returns are time-additive they cannot be added across assets, thus
we need simple returns for portfolio analysis.

    \hypertarget{arma-vs.-arima-models-for-financial-applications}{%
\subsection{1.2 ARMA vs.~ARIMA Models for Financial
Applications}\label{arma-vs.-arima-models-for-financial-applications}}

\hypertarget{arma-vs-arima-suitability}{%
\subsubsection{1.2.1 ARMA vs ARIMA
Suitability}\label{arma-vs-arima-suitability}}

In this section we take the closing prices for the S\&P 500 over the
last 4 years and follow the process in Section 1.1.1 to find the log of
the prices.

    \begin{center}
    \adjustimage{max size={0.9\linewidth}{0.9\paperheight}}{SPMLF_files/SPMLF_21_0.pdf}
    \end{center}
    { \hspace*{\fill} \\}
    
    \begin{center}
    \adjustimage{max size={0.9\linewidth}{0.9\paperheight}}{SPMLF_files/SPMLF_22_0.pdf}
    \end{center}
    { \hspace*{\fill} \\}
    
    As can be seen from the log, sliding window mean and s.d. plots none of
the data is stationary and thus an ARIMA model would be more appropriate
to analyse this data

    \hypertarget{arma10-model}{%
\subsubsection{1.2.2 ARMA(1,0) Model}\label{arma10-model}}

The figure below shows the predicted signal from an ARMA model based on
the data above. In order to better show how the data relates to the
true value, the true value was overlaid on the graph, and only small
window of time is shown. As you can see the model tends to lag behind
the true value, but generally predicts the correct trends. In reality
this day delay would make the model only useful for long term modelling
as any individuals interested in predicting values for a day by day
basis would end up missing daily peaks and troughs. Finally, this model
would also be unable to accommodate for large fluctuations in the market.

    \begin{Verbatim}[commandchars=\\\{\}]
Model Parameters:  [0.99735934]
    \end{Verbatim}

    \begin{center}
    \adjustimage{max size={0.9\linewidth}{0.9\paperheight}}{SPMLF_files/SPMLF_25_1.pdf}
    \end{center}
    { \hspace*{\fill} \\}
    
    \hypertarget{arima110-model}{%
\subsubsection{1.2.3 ARIMA(1,1,0) Model}\label{arima110-model}}

Modelling the same data, now with an ARIMA(1,1,0) model, shows a similar
result. This model doesn't have a large initial prediction like ARMA's,
but does lag behind by a day compared to the true value, rendering it
somewhat unusable for day trading. On the plus side the AR parameter for
this model is -0.00875, which is less than 1 meaning that the prediction
has some stationarity to it.

    \begin{Verbatim}[commandchars=\\\{\}]
Model Parameters:  [-0.0087514]
Avg. Error:  0.0007557235119849103
    \end{Verbatim}

    \begin{center}
    \adjustimage{max size={0.9\linewidth}{0.9\paperheight}}{SPMLF_files/SPMLF_27_1.pdf}
    \end{center}
    { \hspace*{\fill} \\}
    
    \hypertarget{log-prices-and-arima}{%
\subsubsection{1.2.4 Log Prices and ARIMA}\label{log-prices-and-arima}}

To evaluate the neccessity of using log prices when using an ARIMA
model, we will create an ARIMA model using the normal price values and
compare the difference in mean residuals. The figure below shows how the
ARIMA model using normal prices performs. Since the range of data for
the log ARIMA model and normal ARIMA model are vastly different,
comparing the absolute value of the residuals wouldn't yield and helpful
information. Instead, we think of the residuals as the error value, and
divide them by the true value to find the error percentage, this then
allows for a much more meaningful comparison. The ARIMA model based on
the log values has an average error of 0.076\% whereas the ARIMA model
based on the normal values has an average error of 0.58\%, nearly an
order of magnitude higher. Thus, in order to keep residuals much more
accurate we need to use log prices.

    \begin{Verbatim}[commandchars=\\\{\}]
Model Parameters:  [-0.00675003]
Avg. Error:  0.005847536805594489
    \end{Verbatim}

    \begin{center}
    \adjustimage{max size={0.9\linewidth}{0.9\paperheight}}{SPMLF_files/SPMLF_29_1.pdf}
    \end{center}
    { \hspace*{\fill} \\}
    
    \hypertarget{var-models}{%
\subsection{1.3 VAR Models}\label{var-models}}

\hypertarget{section}{%
\subsubsection{1.3.1}\label{section}}

In order to simplify the equation:\\
$(y_t = c + A_1y_{t-1} + A_2y_{t−2} + \cdots + A_py_{t−p} + e_t)$ \\
into\\
\(Y = BZ + U\)\\
We must reason what the four matrices must be.

\(Y\) is a list of all outputs for all variables at all-time points:\\
\(Y = \begin{bmatrix} y_t & y_{t+1} & \cdots & y_{t+T} \end{bmatrix} \in \mathbb{R}^{K\times T}\)\\
Each \(y_t\) is a vector of size \(K\), which is equal to the number of
variables.

\(B\) is a matrix containing all the coefficients represented as
\(A\) in the original equation along with the constant \(c\):\\
\(B = \begin{bmatrix} c & A_1 & A_2 & \cdots & A_p \end{bmatrix} \in \mathbb{R}^{K \times (KP + 1)}\)

\(Z\) contains all the values of \(y\) at different time points:\\
\(Z = \begin{bmatrix}  1 & 1 & \cdots & 1 \\  y_{t-1} & y_{t} & \cdots & y_{t-1+T} \\  y_{t-2} & y_{t-1} & \cdots & y_{t-2+T} \\  \vdots & \vdots & \ddots & \vdots \\  y_{t-p} & y_{t-p+1} & \cdots & y_{t-p+T} \\ \end{bmatrix} \in \mathbb{R}^{(KP + 1) \times T}\)

Finally, \(U\) contains all the error terms represented by \(e_t\) in
the original equation:\\
\(U = \begin{bmatrix} e_t & e_{t+1} & \cdots & e_{t+T} \end{bmatrix} \in \mathbb{R}^{K\times T}\)

Thus we can simply to the concise matrix form: \(Y = BZ + U\).\\
Where:

\(Y \in \mathbb{R}^{K\times 1}\)

\(B \in \mathbb{R}^{K \times (KP + 1)}\)

\(Z \in \mathbb{R}^{(KP + 1) \times T}\)

\(U \in \mathbb{R}^{K\times 1}\)

    \hypertarget{optimal-coefficient}{%
\subsubsection{1.3.2 Optimal Coefficient}\label{optimal-coefficient}}

To find the solution to \(Y = BZ + U\), we can use a multi-variate
least-squares method and attempt to minimize \(U\) which contains the error
terms. Thus, we rearrange and minimize \(UU^T\)

\(min\:UU^T = (Y-BZ)(Y-BZ)^T\)\\
\(min\:UU^T = YY^2 - 2YZ^TB^T + BZZ^TB^T\)\\
\(\frac{\partial UU^T}{\partial B} = 2BZZ^T - 2YZ^T = 0\)\\
$ BZZ^T = YZ^T $\\
thus the optimal value for \(B\) is:\\
\(B = YZ^T(ZZ^T)^{-1}\)

    \hypertarget{eigenvalues-of-a-var}{%
\subsubsection{1.3.3 Eigenvalues of a VAR}\label{eigenvalues-of-a-var}}

A simple VAR(1) process can be written as:\\
\(y_t = Ay_{t-1} + e_t\)\\
By expanding \(y_{t-1}\), \(p\) times recursively we can represent the
equation as:\\
\(y_t = A^py_{t_p} + \sum_i^{p-1}A^ie_{t-i}\)\\
Since it follows that \(A^px = \lambda^px\) where \(\lambda\) is the
eigenvalue of \(A\), if an eigenvalue of A is larger than 1 then \(y\)
will very quickly go to infinity

\hypertarget{portfolio-optimisation}{%
\subsubsection{1.3.4 Portfolio
Optimisation}\label{portfolio-optimisation}}

In this section we investigate method of analysing portfolios to see if
they are diversified. In the first part we shall take five stocks from
the SNP(CAG, MAR, LIN, HCP and MAT) and fit a VAR(1) model on them. The
figures below show the values of the stocks between 2015 and 2019,
as well as the detrended stock values which essentially only plots the
variance around a 66 rolling day mean. LIN only started trading in mid
2018 and thus all the detrended data starts from that point as well.


    \begin{center}
    \adjustimage{max size={0.9\linewidth}{0.9\paperheight}}{SPMLF_files/SPMLF_35_0.pdf}
    \end{center}
    { \hspace*{\fill} \\}
    
            \begin{tcolorbox}[breakable, size=fbox, boxrule=.5pt, pad at break*=1mm, opacityfill=0]
\begin{Verbatim}[commandchars=\\\{\}]
             CAG       MAR       LIN       HCP       MAT
L1.CAG  0.974777  0.039232 -0.123425  0.016729  0.017966
L1.MAR -0.005315  0.911080 -0.050244 -0.011322  0.004972
L1.LIN  0.012038 -0.036269  0.831307 -0.004252 -0.015392
L1.HCP -0.055927  0.030390 -0.243046  0.909823 -0.021972
L1.MAT  0.070240 -0.010227  0.504571  0.031416  0.972620
\end{Verbatim}
\end{tcolorbox}
        
    The table above shows the parameters for the model, or in terms of the
equations we talked about previously, the matrix \(A\). The diagonal has
the highest value across the board, which makes sense because the lagged
value of a single stock has the biggest impact on itself i.e.~the price
of CAG at \(t-1\) has the most impact on its price at \(t\) than all the other stocks. Thus, we can read this matrix as somewhat of a
covariance matrix, where higher magnitudes indicate more dependence
between stocks and the sign represents positive or negative correlation.

One rule of covariance matrices is that the eigenvalues of the matrix
equal the variance across that eigenvector, thus we can deduce that if
the eigenvalues are all similar there is very little correlation across
the entire data.

\hypertarget{eigenvalues-of-portfolio}{%
\paragraph{Eigenvalues of portfolio}\label{eigenvalues-of-portfolio}}

\begin{longtable}[]{@{}lll@{}}
\toprule
Min & Max & Mean\tabularnewline
\midrule
\endhead
0.942 & 0.979 & 0.966\tabularnewline
\bottomrule
\end{longtable}

As can be seen this portfolio is largely uncorrelated and thus can be
considered diversified which is what we want from a portfolio.

    \hypertarget{sector-portfolios}{%
\subsubsection{1.3.5 Sector Portfolios}\label{sector-portfolios}}

Generally speaking having creating a portfolio with only one sector is
ill-advised due to the stocks having similar risks. For example a
legislation on oil companies may cause the stock of all oil companies to
fall. In mathematical terms we would say that sector specific portfolios
are highly correlated. Using a similar methodology from the previous
question we can verify this claim by fitting stocks from a single
sector on a VAR and then finding the eigenvalues of the model
parameters.

The table below show the result of such analysis and as can be seen,
the variance of eigenvalues for most sectors is extremely high, thus
proving the stocks within that portfolio are highly correlated. The
exception to this rule would be Communication Services, Real Estate, and
Energy which seem to be less correlated than the other sectors but still
more than the example portfolio in the previous question.

Thus, if one did want to diversify their portfolio and reduce
correlation across stocks it would be better to choose stocks from
differing industries.

            \begin{tcolorbox}[breakable, size=fbox, boxrule=.5pt, pad at break*=1mm, opacityfill=0]
\begin{Verbatim}[commandchars=\\\{\}]
                       min eigenvalue mean eigenvalue max eigenvalue
Industrials                  0.404429        0.781057       0.984379
Health Care                 0.0221912        0.621437       0.994923
Information Technology       0.437484        0.818534       0.985257
Communication Services       0.758526        0.920201       0.969419
Consumer Discretionary       0.554779        0.849651       0.984175
Utilities                   0.0640217        0.659101       0.986779
Financials                    0.15215        0.636789       0.996305
Materials                   0.0256483        0.688325       0.985673
Real Estate                  0.774359        0.918384       0.975698
Consumer Staples             0.570349        0.863208       0.981215
Energy                       0.836909        0.926981       0.981602
\end{Verbatim}
\end{tcolorbox}
        

    % Add a bibliography block to the postdoc
    
    
    
\end{document}
